\documentclass{beamer}
\usepackage{times}
\usepackage{tikz}
\usepackage{beamerthemesplit}
\usepackage{tcolorbox}

\title{The Security Framework of White-Box Cryptography Revisited}
\author{scnucrypto\inst{1,2}\\ \url{cis.gong@gmail.com}}
\institute{\inst{1}{School of Computer Science, South China Normal University} \\ \inst{2}{Mobile Applications And Security Engineering Center of Guangdong Province}}

\date{\today}

\begin{document}

\frame
{
 \titlepage
}

\section[Outline]{}
\frame{\tableofcontents}

\section{White-box cryptography: background}
\frame{
\frametitle{Cryptography: the very beginning}
"Information theory is about communication in the presence of noise."
\begin{flushright}
-C. Shannon, 1948.\footnote{\scriptsize{\url{http://fab.cba.mit.edu/classes/S62.12/docs/Shannon_noise.pdf}}}
\end{flushright}

\begin{center}
\begin{tikzpicture}
    \node[anchor=south west,inner sep=0] (image) at (0,0) { \includegraphics[width=8cm, height=4cm]{./pics/Shannon_GeneralCommunicationSystem.png}};

    %\begin{scope}[x={(image.south east)},y={(image.north west)}]
        %\draw[help lines,xstep=.1,ystep=.1] (0,0) grid (1,1);
        %\foreach \x in {0,1,...,9} { \node [anchor=north] at (\x/10,0) {0.\x}; }
        %\foreach \y in {0,1,...,9} { \node [anchor=east] at (0,\y/10) {0.\y}; }
        %\draw[green, ultra thick, rounded corners] (0.24,0.18) rectangle (0.50,0.32);
    %\end{scope}
\end{tikzpicture}
\end{center}
}

\frame{
\frametitle{Cryptography: an informal definition}
"Cryptography is about communication in the presence of adversaries."
\begin{flushright}
-R. Rivest
\end{flushright}

\begin{center}
\begin{tikzpicture}
    \node[anchor=south west,inner sep=0] (image) at (0,0) { \includegraphics[width=8cm, height=5cm]{./pics/Shannon_GeneralSecrecySystem.png}};

    %\begin{scope}[x={(image.south east)},y={(image.north west)}]
        %\draw[help lines,xstep=.1,ystep=.1] (0,0) grid (1,1);
        %\foreach \x in {0,1,...,9} { \node [anchor=north] at (\x/10,0) {0.\x}; }
        %\foreach \y in {0,1,...,9} { \node [anchor=east] at (0,\y/10) {0.\y}; }
        %\draw[green, ultra thick, rounded corners] (0.24,0.18) rectangle (0.50,0.32);
    %\end{scope}
\end{tikzpicture}
\end{center}
}

\frame{
\frametitle{Define a secrecy system in ``a mathematically acceptable way"}
\textcolor{red}{``As a first step in the mathematical analysis of cryptography, it is necessary to idealize the situation suitably, and to define in a mathematically acceptable way what we shall mean by a secrecy system."}

\begin{flushright}
-C. Shannon, 1949. \footnote{{\scriptsize``Communication Theory of Secrecy Systems", Bell System Tech. J., vol. 28, pp. 656-715, Oct., 1949.}}
\end{flushright}
}

\frame{
\frametitle{The Kerckhoffs' principle of a secrecy system}
A cipher should be secure when the enemy cryptanalyst knows all details of the enciphering process and deciphering process except for the value of the secret key.

\begin{flushright}
- stated in 1881 by the Dutchman Auguste Kerckhoffs (1835-1903).
\end{flushright}


\begin{center}
\begin{tikzpicture}
    \node[anchor=south west,inner sep=0] (image) at (0,0) { \includegraphics[width=6cm, height=4cm]{./pics/Kerckhoffs.jpg}};

    %\begin{scope}[x={(image.south east)},y={(image.north west)}]
        %\draw[help lines,xstep=.1,ystep=.1] (0,0) grid (1,1);
        %\foreach \x in {0,1,...,9} { \node [anchor=north] at (\x/10,0) {0.\x}; }
        %\foreach \y in {0,1,...,9} { \node [anchor=east] at (0,\y/10) {0.\y}; }
        %\draw[green, ultra thick, rounded corners] (0.24,0.18) rectangle (0.50,0.32);
    %\end{scope}
\end{tikzpicture}
\end{center}
}

\frame{
\frametitle{The goal of modern cryptography}
\begin{itemize}
\item To my opinion, the goal of modern cryptography is to design, analysis, and implement a secrecy system which obtains a mathematically acceptable security proofs.

\begin{itemize}
\item Confidentiality / Secrecy
\item Integrity
\item Authenticity
\item Availability
\end{itemize}

\item Thus provable security must be reduced on \textcolor{red}{a certain model!}
\end{itemize}
}

\frame{
\frametitle{How to modeling attackers in cryptography}
\begin{center}
\begin{tikzpicture}
    %\node[anchor=south west,inner sep=0] (image) at (0,0) { \includegraphics[width=6cm, height=4cm]{./pics/Illustration-of-the-concept-of-Black-White-Grey-box-modeling.png}};
    \node[anchor=south west,inner sep=0] (image) at (0,0) { \includegraphics[width=7cm, height=4cm]{./pics/WBC_Model.png}};

    %\begin{scope}[x={(image.south east)},y={(image.north west)}]
        %\draw[help lines,xstep=.1,ystep=.1] (0,0) grid (1,1);
        %\foreach \x in {0,1,...,9} { \node [anchor=north] at (\x/10,0) {0.\x}; }
        %\foreach \y in {0,1,...,9} { \node [anchor=east] at (0,\y/10) {0.\y}; }
        %\draw[green, ultra thick, rounded corners] (0.24,0.18) rectangle (0.50,0.32);
    %\end{scope}
\end{tikzpicture}
\end{center}
}

\section{White-box cryptography: basic concepts}
\frame{
\frametitle{Black-box model}
The adversary is able to know, choose or adaptively choose inputs and outputs of the function. Given the black-box implementation of the function, the adversary aims to recover the secret values or to misbehave the function.
}

\frame{
\frametitle{White-box model}
\begin{itemize}
\item Informally speaking, an adversary in \textcolor{red}{white-box model} can tamper, modify, manipulate all intermediate values and processes of the implementation of a secrecy system.
\item It can be looked as a superset of \textcolor{red}{black-box} and \textcolor{red}{grey-box} models.
\end{itemize}
}

\frame{
\frametitle{Grey-box model}
According to the black/white-box modeling, the grey-box adversary is able to
\begin{itemize}
\item know, choose or adaptively choose inputs and outputs of the function;

\item tamper, modify, manipulate intermediate values with specific (physical) knowledge.

\end{itemize}
}

\frame{
\frametitle{An illustration of the Black/White/Grey-box modeling}
\begin{center}
\begin{tikzpicture}
    \node[anchor=south west,inner sep=0] (image) at (0,0) { \includegraphics[width=9cm, height=4cm]{./pics/Illustration-of-the-concept-of-Black-White-Grey-box-modeling.png}};
    %\node[anchor=south west,inner sep=0] (image) at (0,0) { \includegraphics[width=7cm, height=4cm]{./pics/WBC_Model.png}};

    %\begin{scope}[x={(image.south east)},y={(image.north west)}]
        %\draw[help lines,xstep=.1,ystep=.1] (0,0) grid (1,1);
        %\foreach \x in {0,1,...,9} { \node [anchor=north] at (\x/10,0) {0.\x}; }
        %\foreach \y in {0,1,...,9} { \node [anchor=east] at (0,\y/10) {0.\y}; }
        %\draw[green, ultra thick, rounded corners] (0.24,0.18) rectangle (0.50,0.32);
    %\end{scope}
\end{tikzpicture}
\end{center}
}


\frame{
\frametitle{The practical perspective of WBC}
Under the definitions of the black-box and the white-box securities, we propose a practical perspective of the white-box cryptography. The perspective is depicted from what the white-box cryptography can provide and how it should be secure against the adversary. Since the secret values are the most important in the white-box security model. A white-box cryptography can provide the following security properties.

\begin{itemize}
\item \textbf{Algorithm level: Key recovery.}

\item \textbf{Module level: Function integrity.}

\item \textbf{System level: Code lifting}

\end{itemize}
}

\frame{
\frametitle{}
Based on the above security properties that white-box cryptography could provide, one can build up secure applications enjoys the white-box security. From the life-cycle of secret keys, we depict the typical usages of the white-box cryptography in the following ways.

\begin{itemize}
\item \textbf{Key distribution.} Normally, the applications need a secure channel to distribute the secret keys between server and client. Although some key distribution schemes can be executed under authenticated channel, but most of them still need the support of certification services (Such as PKI, PKG, etc.). Because the white-box cryptography protects the secret keys from the adversary, the distribution of the keys only requires authenticated channel. If

\item \textbf{Local key management.}

\item \textbf{Dynamic key updating.}

\item \textbf{Key revocation.}
\end{itemize}
}

\frame{

\frametitle{}
For symmetric-key algorithms, they might be used for message authentication, data encryption/decrypiton and pseudorandom number generating. According to these different usages, the security requirements for the implementations of those cryptographic algorithms are diverse. Considering this diversity, we define a practical security notion for white-box cryptography as follows.

\begin{itemize}
\item \textbf{Key integrity.}

\item \textbf{Key confidentiality.}

\item \textbf{Key equivalence.}
\end{itemize}

}


\end{document}
